\documentclass{article}
\usepackage[mathjax]{lwarp}
\usepackage{amsmath}   % For equations
\usepackage{hyperref}  % For internal & external links

\title{Sample \LaTeX{} Document with Links}
\author{Your Name}
\date{\today}

\begin{document}
\maketitle
\tableofcontents

\section{Introduction}
This is a short example demonstrating:
\begin{itemize}
  \item Equations with labels and references,
  \item Internal links (sections, equations),
  \item External links (websites and email).
\end{itemize}

We will define a simple equation in Section~\ref{sec:math} and
refer to it as Equation~\eqref{eq:pythagoras}.

\section{A Simple Equation}
\label{sec:math}

Consider the Pythagorean theorem:
\begin{equation}
  a^2 + b^2 = c^2
  \label{eq:pythagoras}
\end{equation}
We can now reference Equation~\eqref{eq:pythagoras} from anywhere
in the document.

\section{Links}
\subsection{External Links}

Here is a link to the \LaTeX{} project homepage:
\href{https://www.latex-project.org}{\LaTeX{} Project Website}.

You can also typeset a bare URL:
\url{https://www.latex-project.org}.

And an email link:
\href{mailto:someone@example.com}{someone@example.com}

\subsection{Internal Links}
You have already seen internal links via:
\begin{itemize}
  \item The table of contents at the beginning of this document.
  \item Cross-references like Section~\ref{sec:math}
        and Equation~\eqref{eq:pythagoras}.
\end{itemize}

\end{document}
