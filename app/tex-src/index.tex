\documentclass[11pt,a4paper]{article}
\usepackage[mathjax]{lwarp}

% --- Common packages ---
\usepackage[margin=1in]{geometry}   % Page layout
\usepackage{amsmath, amssymb}       % Math
\usepackage{graphicx}               % Images
\usepackage{xcolor}                 % Colors
\usepackage{hyperref}               % Links and PDF metadata

% --- Hyperref setup (make links colored but not too loud) ---
\hypersetup{
    colorlinks=true,
    linkcolor=blue,
    urlcolor=magenta,
    citecolor=teal,
    pdfauthor={Your Name},
    pdftitle={LaTeX Internal and External Links Demo}
}

\title{Demo: Internal \& External Links, Common Packages, and Equations}
\author{Your Name}
\date{\today}

\begin{document}
\maketitle

\tableofcontents
\clearpage

% --------------------------------------------------------
\section{Introduction}
\label{sec:intro}
This document demonstrates:
\begin{itemize}
    \item Internal links to sections (see Section~\ref{sec:internal-links}),
          equations (see Equation~\eqref{eq:pythagoras}),
          and figures (see Figure~\ref{fig:example-figure}).
    \item External links to websites, e.g.~\href{https://www.latex-project.org}{LaTeX Project website}.
    \item Use of common packages like \texttt{amsmath}, \texttt{amssymb}, \texttt{graphicx}, and \texttt{hyperref}.
\end{itemize}

You can also simply show a URL with \verb|\url|:
\url{https://www.ctan.org}

% --------------------------------------------------------
\section{Internal Links}
\label{sec:internal-links}

In this section we add labels to various elements and then reference them.

\subsection{Linking to Sections}
\label{subsec:link-sections}

This is Subsection~\ref{subsec:link-sections}. You can click on its reference
anywhere in the PDF (if your viewer supports hyperlinks). For example, jump back
to the Introduction in Section~\ref{sec:intro}.

\subsection{Linking to Equations}
\label{subsec:link-equations}

Below is an important equation with a label:

\begin{equation}
    a^2 + b^2 = c^2
    \label{eq:pythagoras}
\end{equation}

We can refer to it as Equation~\eqref{eq:pythagoras} anywhere in the text.
This uses \texttt{amsmath} for equation handling.

We can also use other math environments, e.g.~\texttt{align}:

\begin{align}
    \nabla \cdot \vec{E} &= \frac{\rho}{\varepsilon_0}, \label{eq:maxwell1} \\
    \nabla \cdot \vec{B} &= 0,                               \label{eq:maxwell2} \\
    \nabla \times \vec{E} &= -\frac{\partial \vec{B}}{\partial t}, \label{eq:maxwell3} \\
    \nabla \times \vec{B} &= \mu_0 \vec{J} + \mu_0 \varepsilon_0 \frac{\partial \vec{E}}{\partial t}. \label{eq:maxwell4}
\end{align}

Equations~\eqref{eq:maxwell1}--\eqref{eq:maxwell4} show Maxwell's equations.

\subsection{Linking to Figures}
\label{subsec:link-figures}

We can include a figure (placeholder image) and label it for future reference.
If you don't have an image, you can comment out the \verb|\includegraphics| line.

\begin{figure}[h]
    \centering
    % Replace "example-image.png" with your own file, or leave commented.
    % \includegraphics[width=0.5\textwidth]{example-image.png}
    \fbox{\rule{0pt}{2in}\rule{3in}{0pt}} % simple placeholder box
    \caption{Example figure (replace with your image).}
    \label{fig:example-figure}
\end{figure}

Now we can refer to Figure~\ref{fig:example-figure} in the text, and it will be
an internal clickable link.

% --------------------------------------------------------
\section{External Links}
\label{sec:external-links}

External links are usually created with \verb|\href| or \verb|\url| from
the \texttt{hyperref} package.

For example:
\begin{itemize}
    \item \href{https://www.overleaf.com}{Overleaf} -- an online LaTeX editor.
    \item \href{https://www.ctan.org}{CTAN} -- \textbf{C}omprehensive \textbf{T}eX \textbf{A}rchive \textbf{N}etwork.
    \item \url{https://www.github.com} -- shown as plain URL.
\end{itemize}

You can also make external links out of arbitrary text:
\begin{quote}
    For documentation and examples, see the
    \href{https://www.latex-project.org/help/documentation/}{LaTeX documentation page}.
\end{quote}

% --------------------------------------------------------
\section{More Equation Samples (amsmath)}
\label{sec:more-equations}

Here are some additional equation samples using \texttt{amsmath}.

\subsection{Summations and Integrals}

\begin{equation}
    \sum_{n=1}^{\infty} \frac{1}{n^2} = \frac{\pi^2}{6}.
\end{equation}

\begin{equation}
    \int_{0}^{1} x^2 \, dx = \left[\frac{x^3}{3}\right]_0^1 = \frac{1}{3}.
\end{equation}

\subsection{Cases Environment}

\begin{equation}
    f(x) = 
    \begin{cases}
        x^2, & \text{if } x \ge 0, \\
        -x,  & \text{if } x < 0.
    \end{cases}
\end{equation}

\subsection{Matrices}

\begin{equation}
    A = 
    \begin{pmatrix}
        1 & 2 & 3 \\
        0 & -1 & 4 \\
        5 & 2 & 0
    \end{pmatrix}.
\end{equation}

We can mention this matrix again as ``matrix $A$ in Section~\ref{sec:more-equations}''.

% --------------------------------------------------------
\section{Colored Text and Emphasis (xcolor)}
\label{sec:colors}

Using \texttt{xcolor} we can highlight parts of the text:

This is \textcolor{red}{red}, this is \textcolor{blue}{blue},
and this is \textcolor{green!50!black}{dark green}.

We can even combine colors with links, for instance:

\textcolor{blue}{\href{https://en.wikipedia.org/wiki/LaTeX}{Click here to open the LaTeX article on Wikipedia}}.


\end{document}
