\documentclass[11pt]{article}
\usepackage[mathjax]{lwarp}
\usepackage{amsmath,amssymb,amsthm,mathtools}
\usepackage{lipsum}
\usepackage{geometry}
\geometry{margin=1in}

\title{Ongoing Research Index Page}
\author{A. U. Thor}
\date{\today}

\newtheorem{theorem}{Theorem}
\newtheorem{lemma}{Lemma}

\begin{document}
\maketitle

\begin{abstract}
Optionally, use this document to provide optional information.
\end{abstract}

\section{Introduction}
\lipsum[2]

We consider a function $f \colon \mathbb{R} \to \mathbb{R}$.
Basic inline examples include $e^{i\pi} + 1 = 0$ and
$\int_0^1 x^2\,dx = \tfrac{1}{3}$.

\section{Basic Equations}
A numbered equation:
\begin{equation}
  \nabla \cdot \mathbf{E} = \frac{\rho}{\varepsilon_0}.
\end{equation}

An aligned system using \texttt{amsmath}:
\begin{align}
  a^2 + b^2 &= c^2, \label{eq:pythagoras} \\
  (a+b)^2 &= a^2 + 2ab + b^2. \label{eq:binomial}
\end{align}

We also write a general $n$th-degree polynomial:
\begin{equation}
  p(x) = \sum_{k=0}^{n} a_k x^k,
\end{equation}
where $a_k \in \mathbb{R}$ for all $k$.

\section{Piecewise Functions and Matrices}
Using \texttt{cases}:
\begin{equation}
f(x) =
\begin{cases}
  x^2, & x \ge 0, \\
  -x,  & x < 0.
\end{cases}
\end{equation}

A matrix and its determinant:
\begin{equation}
A =
\begin{bmatrix}
  1 & 2 & 3 \\
  0 & 1 & 4 \\
  0 & 0 & 1
\end{bmatrix},
\qquad
\det(A) = 1.
\end{equation}

A small system in augmented-matrix form:
\begin{equation}
\begin{bmatrix}
  1 & 1 & \vline & 2 \\
  2 & -1 & \vline & 0
\end{bmatrix}
\sim
\begin{bmatrix}
  1 & 0 & \vline & \tfrac{2}{3} \\
  0 & 1 & \vline & \tfrac{4}{3}
\end{bmatrix}.
\end{equation}

\section{Theorems and Proof Sketches}
\begin{theorem}[Sample Inequality]
For all $x,y \in \mathbb{R}$,
\[
  (x-y)^2 \ge 0.
\]
\end{theorem}

\begin{proof}
Expanding the square,
\[
  (x-y)^2 = x^2 - 2xy + y^2.
\]
By the rearrangement,
\[
  x^2 - 2xy + y^2 = (x-y)^2,
\]
which is a square of a real number and hence nonnegative.
\end{proof}

\begin{lemma}
Let $x \in \mathbb{R}$. Then
\[
  \cos^2 x + \sin^2 x = 1.
\]
\end{lemma}

\begin{proof}
\lipsum[3]
\end{proof}

\section{More Lorem Ipsum with Inline Math}
\lipsum[4]

Here is some inline math inside lorem ipsum:
\lipsum[5]

We also use \texttt{mathtools} for a nicely aligned relation:
\begin{equation}
  \begin{aligned}
    f(x) &\coloneqq x^3 - 3x + 1, \\
    f'(x) &= 3x^2 - 3, \\
    f''(x) &= 6x.
  \end{aligned}
\end{equation}

\section{Conclusion}
\lipsum[6]

The key relation is summarized as
\begin{equation}
  \lim_{n \to \infty} \left(1 + \frac{1}{n}\right)^n = e.
\end{equation}

\end{document}
